\documentclass[9pt,letterpaper]{article}
\usepackage[left=1.5cm, right=1.5cm, top=2cm]{geometry}
\usepackage{ltablex}
\usepackage{makecell}
\usepackage{tabularx}
\renewcommand\familydefault{\sfdefault}
\usepackage[T1]{fontenc}
\usepackage[usenames, dvipsnames]{color}
\definecolor{parentcolor}{rgb}{0.325, 0.408, 0.584}
\definecolor{modulecolor}{rgb}{1.000, 1.000, 1.000}

\date{}

\renewcommand{\contentsname}{Modules}

\usepackage{hyperref}
\setcounter{tocdepth}{3}
\hypersetup{
    colorlinks=true, %set true if you want colored links
    linktoc=all,     %set to all if you want both sections and subsections linked
    linkcolor=black, %choose some color if you want links to stand out
}

\title{Optohybrid v3 Address Table}
% START: ADDRESS_TABLE_VERSION :: DO NOT EDIT
% END: ADDRESS_TABLE_VERSION :: DO NOT EDIT
\begin{document}

\maketitle
\tableofcontents

% START: ADDRESS_TABLE :: DO NOT EDIT

    \pagebreak
    \section{Module: FPGA.CONTROL \hfill \texttt{0x0}}

    Implements various control and monitoring functions of the Optohybrid\\

    \renewcommand{\arraystretch}{1.3}
    \noindent
    \subsection*{\textcolor{parentcolor}{\textbf{FPGA.CONTROL.LOOPBACK}}}

    \vspace{4mm}
    \noindent
    Loopback data register for testing read/write communication with the Optohybrid FPGA
    \noindent

    \keepXColumns
    \begin{tabularx}{\linewidth}{ | l | l | r | c | l | X | }
    \hline
    \textbf{Node} & \textbf{Adr} & \textbf{Bits} & \textbf{Perm} & \textbf{Def} & \textbf{Description} \\\hline
    \nopagebreak
    DATA & \texttt{0x0} & \texttt{[31:0]} & rw & \texttt{0x1234567} & Write/Read Data Port \\\hline
    \end{tabularx}
    \vspace{5mm}


    \noindent
    \subsection*{\textcolor{parentcolor}{\textbf{FPGA.CONTROL.RELEASE}}}

    \vspace{4mm}
    \noindent
    Optohybrid Firmware Release Date and Version
    \noindent

    \keepXColumns
    \begin{tabularx}{\linewidth}{ | l | l | r | c | l | X | }
    \hline
    \textbf{Node} & \textbf{Adr} & \textbf{Bits} & \textbf{Perm} & \textbf{Def} & \textbf{Description} \\\hline
    \nopagebreak
    DATE & \texttt{0x1} & \texttt{[31:0]} & r & \texttt{} & Release YYYY/MM/DD \\\hline
    \end{tabularx}
    \vspace{5mm}


    \noindent
    \subsection*{\textcolor{parentcolor}{\textbf{FPGA.CONTROL.RELEASE.VERSION}}}

    \vspace{4mm}
    \noindent
    Optohybrid Release Version (XX.YY.ZZ.AA)                                                           \\\\ XX indicates the firmware major version                                                           \\\\ YY indicates the firmware minor version                                                           \\\\ ZZ indicates the firmware patch                                                           \\\\ AA indicates the hardware generation (0C = GE1/1 v3C short, 1C = GE1/1 v3C long, 2A = GE2/1 v1)                                                           
    \noindent

    \keepXColumns
    \begin{tabularx}{\linewidth}{ | l | l | r | c | l | X | }
    \hline
    \textbf{Node} & \textbf{Adr} & \textbf{Bits} & \textbf{Perm} & \textbf{Def} & \textbf{Description} \\\hline
    \nopagebreak
    MAJOR & \texttt{0x2} & \texttt{[7:0]} & r & \texttt{} & Release semantic version major \\\hline
    MINOR & \texttt{0x2} & \texttt{[15:8]} & r & \texttt{} & Release semantic version minor \\\hline
    BUILD & \texttt{0x2} & \texttt{[23:16]} & r & \texttt{} & Release semantic version build \\\hline
    GENERATION & \texttt{0x2} & \texttt{[31:24]} & r & \texttt{} & Release semantic version build \\\hline
    \end{tabularx}
    \vspace{5mm}


    \noindent
    \subsection*{\textcolor{parentcolor}{\textbf{FPGA.CONTROL.SEM}}}

    \vspace{4mm}
    \noindent
    Connects to Outputs of the FPGA's built-in single event upset monitoring system
    \noindent

    \keepXColumns
    \begin{tabularx}{\linewidth}{ | l | l | r | c | l | X | }
    \hline
    \textbf{Node} & \textbf{Adr} & \textbf{Bits} & \textbf{Perm} & \textbf{Def} & \textbf{Description} \\\hline
    \nopagebreak
    CNT\_SEM\_CRITICAL & \texttt{0x3} & \texttt{[15:0]} & r & \texttt{} & Counts of critical single event upsets \\\hline
    CNT\_SEM\_CORRECTION & \texttt{0x4} & \texttt{[31:16]} & r & \texttt{} & Counts of corrected single event upsets \\\hline
    \end{tabularx}
    \vspace{5mm}


    \noindent
    \subsection*{\textcolor{parentcolor}{\textbf{FPGA.CONTROL.VFAT}}}

    \vspace{4mm}
    \noindent
    Controls the 12 VFAT reset outputs from the FPGA
    \noindent

    \keepXColumns
    \begin{tabularx}{\linewidth}{ | l | l | r | c | l | X | }
    \hline
    \textbf{Node} & \textbf{Adr} & \textbf{Bits} & \textbf{Perm} & \textbf{Def} & \textbf{Description} \\\hline
    \nopagebreak
    RESET & \texttt{0x5} & \texttt{[11:0]} & rw & \texttt{0x0} & Mask of VFAT Reset Outputs; 1=reset 0=enable \\\hline
    \end{tabularx}
    \vspace{5mm}


    \noindent
    \subsection*{\textcolor{parentcolor}{\textbf{FPGA.CONTROL.TTC}}}

    \vspace{4mm}
    \noindent
    TTC Status and Control
    \noindent

    \keepXColumns
    \begin{tabularx}{\linewidth}{ | l | l | r | c | l | X | }
    \hline
    \textbf{Node} & \textbf{Adr} & \textbf{Bits} & \textbf{Perm} & \textbf{Def} & \textbf{Description} \\\hline
    \nopagebreak
    BX0\_CNT\_LOCAL & \texttt{0x8} & \texttt{[23:0]} & r & \texttt{} & TTC BX0 Local Counter \\\hline
    BX0\_CNT\_TTC & \texttt{0x9} & \texttt{[23:0]} & r & \texttt{} & TTC BX0 Received Counter \\\hline
    BXN\_CNT\_LOCAL & \texttt{0xa} & \texttt{[11:0]} & r & \texttt{} & TTC BXN Counter \\\hline
    BXN\_SYNC\_ERR & \texttt{0xb} & \texttt{[12:12]} & r & \texttt{} & BXN Synchronization Error; Local BXN and received BXN do not match \\\hline
    BX0\_SYNC\_ERR & \texttt{0xc} & \texttt{[13:13]} & r & \texttt{} & BX0 Synchronization Error \\\hline
    BXN\_OFFSET & \texttt{0xd} & \texttt{[27:16]} & rw & \texttt{0x0} & Local BXN counter offset (starting value at resync) \\\hline
    L1A\_CNT & \texttt{0xe} & \texttt{[23:0]} & r & \texttt{} & L1A Received Counter \\\hline
    BXN\_SYNC\_ERR\_CNT & \texttt{0xf} & \texttt{[15:0]} & r & \texttt{} & BXN Sync Error Counter \\\hline
    BX0\_SYNC\_ERR\_CNT & \texttt{0x10} & \texttt{[31:16]} & r & \texttt{} & BX0 Sync Error Counter \\\hline
    \end{tabularx}
    \vspace{5mm}


    \noindent
    \subsection*{\textcolor{parentcolor}{\textbf{FPGA.CONTROL.SBITS}}}

    \vspace{4mm}
    \noindent
    S-bit and Cluster Packing Rate
    \noindent

    \keepXColumns
    \begin{tabularx}{\linewidth}{ | l | l | r | c | l | X | }
    \hline
    \textbf{Node} & \textbf{Adr} & \textbf{Bits} & \textbf{Perm} & \textbf{Def} & \textbf{Description} \\\hline
    \nopagebreak
    CLUSTER\_RATE & \texttt{0x11} & \texttt{[31:0]} & r & \texttt{} & Trigger cluster rate measured in Hz \\\hline
    \end{tabularx}
    \vspace{5mm}


    \noindent
    \subsection*{\textcolor{parentcolor}{\textbf{FPGA.CONTROL.HDMI}}}

    \vspace{4mm}
    \noindent
    HDMI Connector Control:                                                       \\\\ Mode=0: Each signal is a single VFAT. The VFAT of interest is chosen by SBIT\_SEL                                                       \\\\ Mode=1: Each signal is the OR of three VFATs in an ieta row. The row of interest is configured by SBIT\_SEL                                                       \\\\ Mode=2: Each signal is the OR of four VFATs in an iphi half column (e.g. 0-3, 4-7, 8-11, 12-15, 16-19, 20-23)
    \noindent

    \keepXColumns
    \begin{tabularx}{\linewidth}{ | l | l | r | c | l | X | }
    \hline
    \textbf{Node} & \textbf{Adr} & \textbf{Bits} & \textbf{Perm} & \textbf{Def} & \textbf{Description} \\\hline
    \nopagebreak
    SBIT\_SEL0 & \texttt{0x12} & \texttt{[4:0]} & rw & \texttt{0x0} & HDMI Output 0 S-bit select \\\hline
    SBIT\_SEL1 & \texttt{0x12} & \texttt{[9:5]} & rw & \texttt{0x0} & HDMI Output 1 S-bit select \\\hline
    SBIT\_SEL2 & \texttt{0x12} & \texttt{[14:10]} & rw & \texttt{0x0} & HDMI Output 2 S-bit select \\\hline
    SBIT\_SEL3 & \texttt{0x12} & \texttt{[19:15]} & rw & \texttt{0x0} & HDMI Output 3 S-bit select \\\hline
    SBIT\_SEL4 & \texttt{0x12} & \texttt{[24:20]} & rw & \texttt{0x0} & HDMI Output 4 S-bit select \\\hline
    SBIT\_SEL5 & \texttt{0x12} & \texttt{[29:25]} & rw & \texttt{0x0} & HDMI Output 5 S-bit select \\\hline
    SBIT\_SEL6 & \texttt{0x13} & \texttt{[4:0]} & rw & \texttt{0x0} & HDMI Output 6 S-bit select \\\hline
    SBIT\_SEL7 & \texttt{0x13} & \texttt{[9:5]} & rw & \texttt{0x0} & HDMI Output 7 S-bit select \\\hline
    SBIT\_MODE0 & \texttt{0x13} & \texttt{[11:10]} & rw & \texttt{0x0} & HDMI Output 0 S-bit mode \\\hline
    SBIT\_MODE1 & \texttt{0x13} & \texttt{[13:12]} & rw & \texttt{0x0} & HDMI Output 1 S-bit mode \\\hline
    SBIT\_MODE2 & \texttt{0x13} & \texttt{[15:14]} & rw & \texttt{0x0} & HDMI Output 2 S-bit mode \\\hline
    SBIT\_MODE3 & \texttt{0x13} & \texttt{[17:16]} & rw & \texttt{0x0} & HDMI Output 3 S-bit mode \\\hline
    SBIT\_MODE4 & \texttt{0x13} & \texttt{[19:18]} & rw & \texttt{0x0} & HDMI Output 4 S-bit mode \\\hline
    SBIT\_MODE5 & \texttt{0x13} & \texttt{[21:20]} & rw & \texttt{0x0} & HDMI Output 5 S-bit mode \\\hline
    SBIT\_MODE6 & \texttt{0x13} & \texttt{[23:22]} & rw & \texttt{0x0} & HDMI Output 6 S-bit mode \\\hline
    SBIT\_MODE7 & \texttt{0x13} & \texttt{[25:24]} & rw & \texttt{0x0} & HDMI Output 7 S-bit mode \\\hline
    \end{tabularx}
    \vspace{5mm}


    \noindent
    \subsection*{\textcolor{parentcolor}{\textbf{FPGA.CONTROL.CNT\_SNAP}}}

    \vspace{4mm}
    \noindent
    Control the global counter snapshot
    \noindent

    \keepXColumns
    \begin{tabularx}{\linewidth}{ | l | l | r | c | l | X | }
    \hline
    \textbf{Node} & \textbf{Adr} & \textbf{Bits} & \textbf{Perm} & \textbf{Def} & \textbf{Description} \\\hline
    \nopagebreak
    PULSE & \texttt{0x14} & \texttt{[0:0]} & w & Pulse & Pulse to take a counter snapshot \\\hline
    DISABLE & \texttt{0x15} & \texttt{[1:1]} & rw & \texttt{0x1} & 0=enable snapshots (counters freeze synchronously and need a snapshot to update) \\\hline
    \end{tabularx}
    \vspace{5mm}


    \noindent
    \subsection*{\textcolor{parentcolor}{\textbf{FPGA.CONTROL.DNA}}}

    \vspace{4mm}
    \noindent
    57 Bit FPGA-specific device identifier
    \noindent

    \keepXColumns
    \begin{tabularx}{\linewidth}{ | l | l | r | c | l | X | }
    \hline
    \textbf{Node} & \textbf{Adr} & \textbf{Bits} & \textbf{Perm} & \textbf{Def} & \textbf{Description} \\\hline
    \nopagebreak
    DNA\_LSBS & \texttt{0x17} & \texttt{[31:0]} & r & \texttt{} & Device DNA bits 31 downto 0 \\\hline
    DNA\_MSBS & \texttt{0x18} & \texttt{[24:0]} & r & \texttt{} & Device DNA bits 56 downto 32 \\\hline
    \end{tabularx}
    \vspace{5mm}


    \noindent
    \subsection*{\textcolor{parentcolor}{\textbf{FPGA.CONTROL}}}

    \vspace{4mm}
    \noindent
    Implements various control and monitoring functions of the Optohybrid
    \noindent

    \keepXColumns
    \begin{tabularx}{\linewidth}{ | l | l | r | c | l | X | }
    \hline
    \textbf{Node} & \textbf{Adr} & \textbf{Bits} & \textbf{Perm} & \textbf{Def} & \textbf{Description} \\\hline
    \nopagebreak
    UPTIME & \texttt{0x19} & \texttt{[19:0]} & r & \texttt{} & Uptime in seconds \\\hline
    USR\_ACCESS & \texttt{0x20} & \texttt{[31:0]} & r & \texttt{} & Git hash read from USR\_ACCESS field \\\hline
    \end{tabularx}
    \vspace{5mm}


    \noindent
    \subsection*{\textcolor{parentcolor}{\textbf{FPGA.CONTROL.HOG}}}

    \keepXColumns
    \begin{tabularx}{\linewidth}{ | l | l | r | c | l | X | }
    \hline
    \textbf{Node} & \textbf{Adr} & \textbf{Bits} & \textbf{Perm} & \textbf{Def} & \textbf{Description} \\\hline
    \nopagebreak
    GLOBAL\_DATE & \texttt{0x21} & \texttt{[31:0]} & r & \texttt{} & HOG Global Date \\\hline
    GLOBAL\_TIME & \texttt{0x22} & \texttt{[31:0]} & r & \texttt{} & HOG Global Time \\\hline
    GLOBAL\_VER & \texttt{0x23} & \texttt{[31:0]} & r & \texttt{} & HOG Global Version \\\hline
    GLOBAL\_SHA & \texttt{0x24} & \texttt{[31:0]} & r & \texttt{} & HOG Global SHA \\\hline
    TOP\_SHA & \texttt{0x25} & \texttt{[31:0]} & r & \texttt{} & HOG Top SHA \\\hline
    TOP\_VER & \texttt{0x26} & \texttt{[31:0]} & r & \texttt{} & HOG Top Version \\\hline
    HOG\_SHA & \texttt{0x27} & \texttt{[31:0]} & r & \texttt{} & HOG SHA \\\hline
    HOG\_VER & \texttt{0x28} & \texttt{[31:0]} & r & \texttt{} & HOG Version \\\hline
    OH\_SHA & \texttt{0x29} & \texttt{[31:0]} & r & \texttt{} & OH SHA \\\hline
    OH\_VER & \texttt{0x2a} & \texttt{[31:0]} & r & \texttt{} & OH Version \\\hline
    FLAVOUR & \texttt{0x2b} & \texttt{[31:0]} & r & \texttt{} & Flavor \\\hline
    \end{tabularx}
    \vspace{5mm}



    \pagebreak
    \section{Module: FPGA.ADC \hfill \texttt{0x1000}}

    Connects to the Virtex-6 XADC and allows for reading of temperature, VCCINT, and VCCAUX voltages\\

    \renewcommand{\arraystretch}{1.3}
    \noindent
    \subsection*{\textcolor{parentcolor}{\textbf{FPGA.ADC.CTRL}}}

    \keepXColumns
    \begin{tabularx}{\linewidth}{ | l | l | r | c | l | X | }
    \hline
    \textbf{Node} & \textbf{Adr} & \textbf{Bits} & \textbf{Perm} & \textbf{Def} & \textbf{Description} \\\hline
    \nopagebreak
    OVERTEMP & \texttt{0x1000} & \texttt{[0:0]} & r & \texttt{} & FPGA over temperature \\\hline
    VCCAUX\_ALARM & \texttt{0x1000} & \texttt{[1:1]} & r & \texttt{} & FPGA VCCAUX Alarm \\\hline
    VCCINT\_ALARM & \texttt{0x1000} & \texttt{[2:2]} & r & \texttt{} & FPGA VCCINT Alarm \\\hline
    ADR\_IN & \texttt{0x1000} & \texttt{[9:3]} & rw & \texttt{0x0} & XADC Addr In \\\hline
    ENABLE & \texttt{0x1000} & \texttt{[10:10]} & rw & \texttt{0x1} & XADC Data In \\\hline
    CNT\_OVERTEMP & \texttt{0x1000} & \texttt{[17:11]} & r & \texttt{} & Overtemperature counter \\\hline
    CNT\_VCCAUX\_ALARM & \texttt{0x1000} & \texttt{[24:18]} & r & \texttt{} & VCCAUX Alarm Counter \\\hline
    CNT\_VCCINT\_ALARM & \texttt{0x1000} & \texttt{[31:25]} & r & \texttt{} & VCCINT Alarm Counter \\\hline
    DATA\_IN & \texttt{0x1001} & \texttt{[15:0]} & rw & \texttt{0x0} & XADC Data In \\\hline
    DATA\_OUT & \texttt{0x1001} & \texttt{[31:16]} & r & \texttt{} & XADC Data Out \\\hline
    RESET & \texttt{0x1002} & \texttt{[0:0]} & w & Pulse & XADC Reset \\\hline
    WR\_EN & \texttt{0x1003} & \texttt{[0:0]} & w & Pulse & XADC Write Enable \\\hline
    \end{tabularx}
    \vspace{5mm}



    \pagebreak
    \section{Module: FPGA.TRIG \hfill \texttt{0x2000}}

    Connects to the trigger control module\\

    \renewcommand{\arraystretch}{1.3}
    \noindent
    \subsection*{\textcolor{parentcolor}{\textbf{FPGA.TRIG.CTRL}}}

    \vspace{4mm}
    \noindent
    Controls and monitors various parameters of the S-bit deserialization and cluster building.
    \noindent

    \keepXColumns
    \begin{tabularx}{\linewidth}{ | l | l | r | c | l | X | }
    \hline
    \textbf{Node} & \textbf{Adr} & \textbf{Bits} & \textbf{Perm} & \textbf{Def} & \textbf{Description} \\\hline
    \nopagebreak
    VFAT\_MASK & \texttt{0x2000} & \texttt{[11:0]} & rw & \texttt{0x0} & 12 bit mask of VFATs (1=off) \\\hline
    SBIT\_DEADTIME & \texttt{0x2000} & \texttt{[27:24]} & rw & \texttt{0x7} & Set programmable oneshot deadtime which applies to retriggers on individual VFAT channels \\\hline
    ACTIVE\_VFATS & \texttt{0x2001} & \texttt{[11:0]} & r & \texttt{} & 12 bit list of VFATs with hits in this BX \\\hline
    CNT\_OVERFLOW & \texttt{0x2002} & \texttt{[15:0]} & r & \texttt{} & Overflow Counter (more than 8 clusters in a bx) \\\hline
    ALIGNED\_COUNT\_TO\_READY & \texttt{0x2002} & \texttt{[27:16]} & rw & \texttt{0x1FF} & Number of link consecutive good frames required before the transmission unit is marked as good and S-bits can be produced \\\hline
    SBIT\_SOT\_READY & \texttt{0x2003} & \texttt{[11:0]} & r & \texttt{} & 12 bit list of VFATs with stable Start-of-frame pulses (in sync for a number of clock cycles) \\\hline
    SBIT\_SOT\_UNSTABLE & \texttt{0x2004} & \texttt{[11:0]} & r & \texttt{} & 12 bit list of VFATs with unstable Start-of-frame pulses (became misaligned after already achieving lock) \\\hline
    \end{tabularx}
    \vspace{5mm}


    \noindent
    \subsection*{\textcolor{parentcolor}{\textbf{FPGA.TRIG.CTRL.SBITS\_MUX}}}

    \vspace{4mm}
    \noindent
    Multiplexed copy of Sbits from a selected VFAT
    \noindent

    \keepXColumns
    \begin{tabularx}{\linewidth}{ | l | l | r | c | l | X | }
    \hline
    \textbf{Node} & \textbf{Adr} & \textbf{Bits} & \textbf{Perm} & \textbf{Def} & \textbf{Description} \\\hline
    \nopagebreak
    SBIT\_MUX\_SEL & \texttt{0x200e} & \texttt{[8:4]} & rw & \texttt{0x10} & Select a VFAT which will connect to the S-bit multiplexer \\\hline
    SBITS\_MUX\_LSB & \texttt{0x200f} & \texttt{[31:0]} & r & \texttt{} & Multiplexed S-bits 31 to 0 \\\hline
    SBITS\_MUX\_MSB & \texttt{0x2010} & \texttt{[31:0]} & r & \texttt{} & Multiplexed S-bits 63 to 32 \\\hline
    \end{tabularx}
    \vspace{5mm}


    \noindent
    \subsection*{\textcolor{parentcolor}{\textbf{FPGA.TRIG.CNT}}}

    \vspace{4mm}
    \noindent
    S-BIT Counters \\\\  Set CNT\_PERSIST to 1 to accumulate. Otherwise the counters will automatically reset after a programmable time (default is 1 second). By default this time is 1 second, making these counters a rate counter in Hertz
    \noindent

    \keepXColumns
    \begin{tabularx}{\linewidth}{ | l | l | r | c | l | X | }
    \hline
    \textbf{Node} & \textbf{Adr} & \textbf{Bits} & \textbf{Perm} & \textbf{Def} & \textbf{Description} \\\hline
    \nopagebreak
    VFAT0\_SBITS & \texttt{0x2017} & \texttt{[31:0]} & r & \texttt{} & VFAT 0 Counter \\\hline
    VFAT1\_SBITS & \texttt{0x2018} & \texttt{[31:0]} & r & \texttt{} & VFAT 1 Counter \\\hline
    VFAT2\_SBITS & \texttt{0x2019} & \texttt{[31:0]} & r & \texttt{} & VFAT 2 Counter \\\hline
    VFAT3\_SBITS & \texttt{0x201a} & \texttt{[31:0]} & r & \texttt{} & VFAT 3 Counter \\\hline
    VFAT4\_SBITS & \texttt{0x201b} & \texttt{[31:0]} & r & \texttt{} & VFAT 4 Counter \\\hline
    VFAT5\_SBITS & \texttt{0x201c} & \texttt{[31:0]} & r & \texttt{} & VFAT 5 Counter \\\hline
    VFAT6\_SBITS & \texttt{0x201d} & \texttt{[31:0]} & r & \texttt{} & VFAT 6 Counter \\\hline
    VFAT7\_SBITS & \texttt{0x201e} & \texttt{[31:0]} & r & \texttt{} & VFAT 7 Counter \\\hline
    VFAT8\_SBITS & \texttt{0x201f} & \texttt{[31:0]} & r & \texttt{} & VFAT 8 Counter \\\hline
    VFAT9\_SBITS & \texttt{0x2020} & \texttt{[31:0]} & r & \texttt{} & VFAT 9 Counter \\\hline
    VFAT10\_SBITS & \texttt{0x2021} & \texttt{[31:0]} & r & \texttt{} & VFAT 10 Counter \\\hline
    VFAT11\_SBITS & \texttt{0x2022} & \texttt{[31:0]} & r & \texttt{} & VFAT 11 Counter \\\hline
    RESET & \texttt{0x202f} & \texttt{[0:0]} & w & Pulse & Reset S-bit counters \\\hline
    SBIT\_CNT\_PERSIST & \texttt{0x2030} & \texttt{[0:0]} & rw & \texttt{0x0} & 1=counters will persist until manually reset; \\ & & & & &                                0=counters will automatically reset at CNT\_TIME \\\hline
    SBIT\_CNT\_TIME\_MAX & \texttt{0x2031} & \texttt{[31:0]} & rw & \texttt{0x2638E98} & Number of BX that the VFAT S-bit counters will count to before automatically resetting to zero \\\hline
    CLUSTER\_COUNT & \texttt{0x2032} & \texttt{[31:0]} & r & \texttt{} & VFAT Cluster Counter (chamber) \\\hline
    SBITS\_OVER\_64x0 & \texttt{0x2036} & \texttt{[15:0]} & r & \texttt{} & More than 64 * 0 Sbits in a bx Counter \\\hline
    SBITS\_OVER\_64x1 & \texttt{0x2037} & \texttt{[15:0]} & r & \texttt{} & More than 64 * 1 Sbits in a bx Counter \\\hline
    SBITS\_OVER\_64x2 & \texttt{0x2038} & \texttt{[15:0]} & r & \texttt{} & More than 64 * 2 Sbits in a bx Counter \\\hline
    SBITS\_OVER\_64x3 & \texttt{0x2039} & \texttt{[15:0]} & r & \texttt{} & More than 64 * 3 Sbits in a bx Counter \\\hline
    SBITS\_OVER\_64x4 & \texttt{0x203a} & \texttt{[15:0]} & r & \texttt{} & More than 64 * 4 Sbits in a bx Counter \\\hline
    SBITS\_OVER\_64x5 & \texttt{0x203b} & \texttt{[15:0]} & r & \texttt{} & More than 64 * 5 Sbits in a bx Counter \\\hline
    SBITS\_OVER\_64x6 & \texttt{0x203c} & \texttt{[15:0]} & r & \texttt{} & More than 64 * 6 Sbits in a bx Counter \\\hline
    SBITS\_OVER\_64x7 & \texttt{0x203d} & \texttt{[15:0]} & r & \texttt{} & More than 64 * 7 Sbits in a bx Counter \\\hline
    SBITS\_OVER\_64x8 & \texttt{0x203e} & \texttt{[15:0]} & r & \texttt{} & More than 64 * 8 Sbits in a bx Counter \\\hline
    SBITS\_OVER\_64x9 & \texttt{0x203f} & \texttt{[15:0]} & r & \texttt{} & More than 64 * 9 Sbits in a bx Counter \\\hline
    SBITS\_OVER\_64x10 & \texttt{0x2040} & \texttt{[15:0]} & r & \texttt{} & More than 64 * 10 Sbits in a bx Counter \\\hline
    SBITS\_OVER\_64x11 & \texttt{0x2041} & \texttt{[15:0]} & r & \texttt{} & More than 64 * 11 Sbits in a bx Counter \\\hline
    \end{tabularx}
    \vspace{5mm}


    \noindent
    \subsection*{\textcolor{parentcolor}{\textbf{FPGA.TRIG.SBIT\_MONITOR}}}

    \vspace{4mm}
    \noindent
    sbit monitor module which shows the first valid sbit clusters after a reset on the selected link
    \noindent

    \keepXColumns
    \begin{tabularx}{\linewidth}{ | l | l | r | c | l | X | }
    \hline
    \textbf{Node} & \textbf{Adr} & \textbf{Bits} & \textbf{Perm} & \textbf{Def} & \textbf{Description} \\\hline
    \nopagebreak
    RESET & \texttt{0x2090} & \texttt{[31:0]} & w & Pulse & Reset the sbit monitor module and re-arm for triggering \\\hline
    CLUSTER0 & \texttt{0x2091} & \texttt{[15:0]} & r & \texttt{} & Last cluster 0 \\\hline
    CLUSTER1 & \texttt{0x2092} & \texttt{[15:0]} & r & \texttt{} & Last cluster 1 \\\hline
    CLUSTER2 & \texttt{0x2093} & \texttt{[15:0]} & r & \texttt{} & Last cluster 2 \\\hline
    CLUSTER3 & \texttt{0x2094} & \texttt{[15:0]} & r & \texttt{} & Last cluster 3 \\\hline
    CLUSTER4 & \texttt{0x2095} & \texttt{[15:0]} & r & \texttt{} & Last cluster 4 \\\hline
    CLUSTER5 & \texttt{0x2096} & \texttt{[15:0]} & r & \texttt{} & Last cluster 5 \\\hline
    CLUSTER6 & \texttt{0x2097} & \texttt{[15:0]} & r & \texttt{} & Last cluster 6 \\\hline
    CLUSTER7 & \texttt{0x2098} & \texttt{[15:0]} & r & \texttt{} & Last cluster 7 \\\hline
    L1A\_DELAY & \texttt{0x20a0} & \texttt{[31:0]} & r & \texttt{} & Number of BX between this sbit and the subsequent L1A \\\hline
    \end{tabularx}
    \vspace{5mm}


    \noindent
    \subsection*{\textcolor{parentcolor}{\textbf{FPGA.TRIG.SBIT\_HITMAP}}}

    \vspace{4mm}
    \noindent
    The Sbit hitmap module accumulates all incoming Sbits during a period of time
    \noindent

    \keepXColumns
    \begin{tabularx}{\linewidth}{ | l | l | r | c | l | X | }
    \hline
    \textbf{Node} & \textbf{Adr} & \textbf{Bits} & \textbf{Perm} & \textbf{Def} & \textbf{Description} \\\hline
    \nopagebreak
    RESET & \texttt{0x20b0} & \texttt{[31:0]} & w & Pulse & Reset the accumulation registers \\\hline
    ACQUIRE & \texttt{0x20b1} & \texttt{[0:0]} & rw & \texttt{0x0} & Sbits are accumulated as long as this flag is set \\\hline
    VFAT0\_MSB & \texttt{0x20b2} & \texttt{[31:0]} & r & \texttt{} & Accumulator for Sbit 63 to 32 of VFAT0 \\\hline
    VFAT0\_LSB & \texttt{0x20b3} & \texttt{[31:0]} & r & \texttt{} & Accumulator for Sbit 31 to 0 of VFAT0 \\\hline
    VFAT1\_MSB & \texttt{0x20b4} & \texttt{[31:0]} & r & \texttt{} & Accumulator for Sbit 63 to 32 of VFAT1 \\\hline
    VFAT1\_LSB & \texttt{0x20b5} & \texttt{[31:0]} & r & \texttt{} & Accumulator for Sbit 31 to 0 of VFAT1 \\\hline
    VFAT2\_MSB & \texttt{0x20b6} & \texttt{[31:0]} & r & \texttt{} & Accumulator for Sbit 63 to 32 of VFAT2 \\\hline
    VFAT2\_LSB & \texttt{0x20b7} & \texttt{[31:0]} & r & \texttt{} & Accumulator for Sbit 31 to 0 of VFAT2 \\\hline
    VFAT3\_MSB & \texttt{0x20b8} & \texttt{[31:0]} & r & \texttt{} & Accumulator for Sbit 63 to 32 of VFAT3 \\\hline
    VFAT3\_LSB & \texttt{0x20b9} & \texttt{[31:0]} & r & \texttt{} & Accumulator for Sbit 31 to 0 of VFAT3 \\\hline
    VFAT4\_MSB & \texttt{0x20ba} & \texttt{[31:0]} & r & \texttt{} & Accumulator for Sbit 63 to 32 of VFAT4 \\\hline
    VFAT4\_LSB & \texttt{0x20bb} & \texttt{[31:0]} & r & \texttt{} & Accumulator for Sbit 31 to 0 of VFAT4 \\\hline
    VFAT5\_MSB & \texttt{0x20bc} & \texttt{[31:0]} & r & \texttt{} & Accumulator for Sbit 63 to 32 of VFAT5 \\\hline
    VFAT5\_LSB & \texttt{0x20bd} & \texttt{[31:0]} & r & \texttt{} & Accumulator for Sbit 31 to 0 of VFAT5 \\\hline
    VFAT6\_MSB & \texttt{0x20be} & \texttt{[31:0]} & r & \texttt{} & Accumulator for Sbit 63 to 32 of VFAT6 \\\hline
    VFAT6\_LSB & \texttt{0x20bf} & \texttt{[31:0]} & r & \texttt{} & Accumulator for Sbit 31 to 0 of VFAT6 \\\hline
    VFAT7\_MSB & \texttt{0x20c0} & \texttt{[31:0]} & r & \texttt{} & Accumulator for Sbit 63 to 32 of VFAT7 \\\hline
    VFAT7\_LSB & \texttt{0x20c1} & \texttt{[31:0]} & r & \texttt{} & Accumulator for Sbit 31 to 0 of VFAT7 \\\hline
    VFAT8\_MSB & \texttt{0x20c2} & \texttt{[31:0]} & r & \texttt{} & Accumulator for Sbit 63 to 32 of VFAT8 \\\hline
    VFAT8\_LSB & \texttt{0x20c3} & \texttt{[31:0]} & r & \texttt{} & Accumulator for Sbit 31 to 0 of VFAT8 \\\hline
    VFAT9\_MSB & \texttt{0x20c4} & \texttt{[31:0]} & r & \texttt{} & Accumulator for Sbit 63 to 32 of VFAT9 \\\hline
    VFAT9\_LSB & \texttt{0x20c5} & \texttt{[31:0]} & r & \texttt{} & Accumulator for Sbit 31 to 0 of VFAT9 \\\hline
    VFAT10\_MSB & \texttt{0x20c6} & \texttt{[31:0]} & r & \texttt{} & Accumulator for Sbit 63 to 32 of VFAT10 \\\hline
    VFAT10\_LSB & \texttt{0x20c7} & \texttt{[31:0]} & r & \texttt{} & Accumulator for Sbit 31 to 0 of VFAT10 \\\hline
    VFAT11\_MSB & \texttt{0x20c8} & \texttt{[31:0]} & r & \texttt{} & Accumulator for Sbit 63 to 32 of VFAT11 \\\hline
    VFAT11\_LSB & \texttt{0x20c9} & \texttt{[31:0]} & r & \texttt{} & Accumulator for Sbit 31 to 0 of VFAT11 \\\hline
    \end{tabularx}
    \vspace{5mm}


    \noindent
    \subsection*{\textcolor{parentcolor}{\textbf{FPGA.TRIG.CTRL}}}

    \vspace{4mm}
    \noindent
    Controls and monitors various parameters of the S-bit deserialization and cluster building.
    \noindent

    \keepXColumns
    \begin{tabularx}{\linewidth}{ | l | l | r | c | l | X | }
    \hline
    \textbf{Node} & \textbf{Adr} & \textbf{Bits} & \textbf{Perm} & \textbf{Def} & \textbf{Description} \\\hline
    \nopagebreak
    SBIT\_SOT\_INVALID\_BITSKIP & \texttt{0x20e2} & \texttt{[11:0]} & r & \texttt{} & 12 bit list of VFATs with a invalid bitskip counter for Start-of-frame pulses \\\hline
    \end{tabularx}
    \vspace{5mm}


% END: ADDRESS_TABLE :: DO NOT EDIT

\end{document}
