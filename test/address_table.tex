\documentclass[9pt,letterpaper]{article}
\usepackage[left=1.5cm, right=1.5cm, top=2cm]{geometry}
\usepackage{ltablex}
\usepackage{makecell}
\usepackage{tabularx}
\renewcommand\familydefault{\sfdefault}
\usepackage[T1]{fontenc}
\usepackage[usenames, dvipsnames]{color}
\definecolor{parentcolor}{rgb}{0.325, 0.408, 0.584}
\definecolor{modulecolor}{rgb}{1.000, 1.000, 1.000}

\date{}

\renewcommand{\contentsname}{Modules}

\usepackage{hyperref}
\setcounter{tocdepth}{3}
\hypersetup{
    colorlinks=true, %set true if you want colored links
    linktoc=all,     %set to all if you want both sections and subsections linked
    linkcolor=black, %choose some color if you want links to stand out
}

\title{Optohybrid v3 Address Table}
% START: ADDRESS_TABLE_VERSION :: DO NOT EDIT
% END: ADDRESS_TABLE_VERSION :: DO NOT EDIT
\begin{document}

\maketitle
\tableofcontents

% START: ADDRESS_TABLE :: DO NOT EDIT

    \pagebreak
    \section{Module: FPGA.CONTROL \hfill \texttt{0x0}}

    Implements various control and monitoring functions of the Optohybrid\\

    \renewcommand{\arraystretch}{1.3}
    \noindent
    \subsection*{\textcolor{parentcolor}{\textbf{FPGA.CONTROL.LOOPBACK}}}

    \vspace{4mm}
    \noindent
    Loopback data register for testing read/write communication with the Optohybrid FPGA
    \noindent

    \keepXColumns
    \begin{tabularx}{\linewidth}{ | l | l | r | c | l | X | }
    \hline
    \textbf{Node} & \textbf{Adr} & \textbf{Bits} & \textbf{Perm} & \textbf{Def} & \textbf{Description} \\\hline
    \nopagebreak
    DATA & \texttt{0x0} & \texttt{[31:0]} & rw & \texttt{0x1234567} & Write/Read Data Port \\\hline
    \end{tabularx}
    \vspace{5mm}


    \noindent
    \subsection*{\textcolor{parentcolor}{\textbf{FPGA.CONTROL.RELEASE}}}

    \vspace{4mm}
    \noindent
    Optohybrid Firmware Release Date and Version
    \noindent

    \keepXColumns
    \begin{tabularx}{\linewidth}{ | l | l | r | c | l | X | }
    \hline
    \textbf{Node} & \textbf{Adr} & \textbf{Bits} & \textbf{Perm} & \textbf{Def} & \textbf{Description} \\\hline
    \nopagebreak
    DATE & \texttt{0x1} & \texttt{[31:0]} & r & \texttt{} & Release YYYY/MM/DD \\\hline
    \end{tabularx}
    \vspace{5mm}


    \noindent
    \subsection*{\textcolor{parentcolor}{\textbf{FPGA.CONTROL.RELEASE.VERSION}}}

    \vspace{4mm}
    \noindent
    Optohybrid Release Version (XX.YY.ZZ.AA)                                                           \\\\ XX indicates the firmware major version                                                           \\\\ YY indicates the firmware minor version                                                           \\\\ ZZ indicates the firmware patch                                                           \\\\ AA indicates the hardware generation (0C = GE1/1 v3C short, 1C = GE1/1 v3C long, 2A = GE2/1 v1)                                                           
    \noindent

    \keepXColumns
    \begin{tabularx}{\linewidth}{ | l | l | r | c | l | X | }
    \hline
    \textbf{Node} & \textbf{Adr} & \textbf{Bits} & \textbf{Perm} & \textbf{Def} & \textbf{Description} \\\hline
    \nopagebreak
    MAJOR & \texttt{0x2} & \texttt{[7:0]} & r & \texttt{} & Release semantic version major \\\hline
    MINOR & \texttt{0x2} & \texttt{[15:8]} & r & \texttt{} & Release semantic version minor \\\hline
    BUILD & \texttt{0x2} & \texttt{[23:16]} & r & \texttt{} & Release semantic version build \\\hline
    GENERATION & \texttt{0x2} & \texttt{[31:24]} & r & \texttt{} & Release semantic version build \\\hline
    \end{tabularx}
    \vspace{5mm}


    \noindent
    \subsection*{\textcolor{parentcolor}{\textbf{FPGA.CONTROL.SEM}}}

    \vspace{4mm}
    \noindent
    Connects to Outputs of the FPGA's built-in single event upset monitoring system
    \noindent

    \keepXColumns
    \begin{tabularx}{\linewidth}{ | l | l | r | c | l | X | }
    \hline
    \textbf{Node} & \textbf{Adr} & \textbf{Bits} & \textbf{Perm} & \textbf{Def} & \textbf{Description} \\\hline
    \nopagebreak
    CNT\_SEM\_CRITICAL & \texttt{0x3} & \texttt{[15:0]} & r & \texttt{} & Counts of critical single event upsets \\\hline
    CNT\_SEM\_CORRECTION & \texttt{0x4} & \texttt{[31:16]} & r & \texttt{} & Counts of corrected single event upsets \\\hline
    \end{tabularx}
    \vspace{5mm}


    \noindent
    \subsection*{\textcolor{parentcolor}{\textbf{FPGA.CONTROL.VFAT}}}

    \vspace{4mm}
    \noindent
    Controls the 12 VFAT reset outputs from the FPGA
    \noindent

    \keepXColumns
    \begin{tabularx}{\linewidth}{ | l | l | r | c | l | X | }
    \hline
    \textbf{Node} & \textbf{Adr} & \textbf{Bits} & \textbf{Perm} & \textbf{Def} & \textbf{Description} \\\hline
    \nopagebreak
    RESET & \texttt{0x5} & \texttt{[11:0]} & rw & \texttt{0x0} & Mask of VFAT Reset Outputs; 1=reset 0=enable \\\hline
    \end{tabularx}
    \vspace{5mm}


    \noindent
    \subsection*{\textcolor{parentcolor}{\textbf{FPGA.CONTROL.TTC}}}

    \vspace{4mm}
    \noindent
    TTC Status and Control
    \noindent

    \keepXColumns
    \begin{tabularx}{\linewidth}{ | l | l | r | c | l | X | }
    \hline
    \textbf{Node} & \textbf{Adr} & \textbf{Bits} & \textbf{Perm} & \textbf{Def} & \textbf{Description} \\\hline
    \nopagebreak
    BX0\_CNT\_LOCAL & \texttt{0x8} & \texttt{[23:0]} & r & \texttt{} & TTC BX0 Local Counter \\\hline
    BX0\_CNT\_TTC & \texttt{0x9} & \texttt{[23:0]} & r & \texttt{} & TTC BX0 Received Counter \\\hline
    BXN\_CNT\_LOCAL & \texttt{0xa} & \texttt{[11:0]} & r & \texttt{} & TTC BXN Counter \\\hline
    BXN\_SYNC\_ERR & \texttt{0xb} & \texttt{[12:12]} & r & \texttt{} & BXN Synchronization Error; Local BXN and received BXN do not match \\\hline
    BX0\_SYNC\_ERR & \texttt{0xc} & \texttt{[13:13]} & r & \texttt{} & BX0 Synchronization Error \\\hline
    BXN\_OFFSET & \texttt{0xd} & \texttt{[27:16]} & rw & \texttt{0x0} & Local BXN counter offset (starting value at resync) \\\hline
    L1A\_CNT & \texttt{0xe} & \texttt{[23:0]} & r & \texttt{} & L1A Received Counter \\\hline
    BXN\_SYNC\_ERR\_CNT & \texttt{0xf} & \texttt{[15:0]} & r & \texttt{} & BXN Sync Error Counter \\\hline
    BX0\_SYNC\_ERR\_CNT & \texttt{0x10} & \texttt{[31:16]} & r & \texttt{} & BX0 Sync Error Counter \\\hline
    \end{tabularx}
    \vspace{5mm}


    \noindent
    \subsection*{\textcolor{parentcolor}{\textbf{FPGA.CONTROL.SBITS}}}

    \vspace{4mm}
    \noindent
    S-bit and Cluster Packing Rate
    \noindent

    \keepXColumns
    \begin{tabularx}{\linewidth}{ | l | l | r | c | l | X | }
    \hline
    \textbf{Node} & \textbf{Adr} & \textbf{Bits} & \textbf{Perm} & \textbf{Def} & \textbf{Description} \\\hline
    \nopagebreak
    CLUSTER\_RATE & \texttt{0x11} & \texttt{[31:0]} & r & \texttt{} & Trigger cluster rate measured in Hz \\\hline
    \end{tabularx}
    \vspace{5mm}


    \noindent
    \subsection*{\textcolor{parentcolor}{\textbf{FPGA.CONTROL.HDMI}}}

    \vspace{4mm}
    \noindent
    HDMI Connector Control:                                                       \\\\ Mode=0: Each signal is a single VFAT. The VFAT of interest is chosen by SBIT\_SEL                                                       \\\\ Mode=1: Each signal is the OR of three VFATs in an ieta row. The row of interest is configured by SBIT\_SEL                                                       \\\\ Mode=2: Each signal is the OR of four VFATs in an iphi half column (e.g. 0-3, 4-7, 8-11, 12-15, 16-19, 20-23)
    \noindent

    \keepXColumns
    \begin{tabularx}{\linewidth}{ | l | l | r | c | l | X | }
    \hline
    \textbf{Node} & \textbf{Adr} & \textbf{Bits} & \textbf{Perm} & \textbf{Def} & \textbf{Description} \\\hline
    \nopagebreak
    SBIT\_SEL0 & \texttt{0x12} & \texttt{[4:0]} & rw & \texttt{0x0} & HDMI Output 0 S-bit select \\\hline
    SBIT\_SEL1 & \texttt{0x12} & \texttt{[9:5]} & rw & \texttt{0x0} & HDMI Output 1 S-bit select \\\hline
    SBIT\_SEL2 & \texttt{0x12} & \texttt{[14:10]} & rw & \texttt{0x0} & HDMI Output 2 S-bit select \\\hline
    SBIT\_SEL3 & \texttt{0x12} & \texttt{[19:15]} & rw & \texttt{0x0} & HDMI Output 3 S-bit select \\\hline
    SBIT\_SEL4 & \texttt{0x12} & \texttt{[24:20]} & rw & \texttt{0x0} & HDMI Output 4 S-bit select \\\hline
    SBIT\_SEL5 & \texttt{0x12} & \texttt{[29:25]} & rw & \texttt{0x0} & HDMI Output 5 S-bit select \\\hline
    SBIT\_SEL6 & \texttt{0x13} & \texttt{[4:0]} & rw & \texttt{0x0} & HDMI Output 6 S-bit select \\\hline
    SBIT\_SEL7 & \texttt{0x13} & \texttt{[9:5]} & rw & \texttt{0x0} & HDMI Output 7 S-bit select \\\hline
    SBIT\_MODE0 & \texttt{0x13} & \texttt{[11:10]} & rw & \texttt{0x0} & HDMI Output 0 S-bit mode \\\hline
    SBIT\_MODE1 & \texttt{0x13} & \texttt{[13:12]} & rw & \texttt{0x0} & HDMI Output 1 S-bit mode \\\hline
    SBIT\_MODE2 & \texttt{0x13} & \texttt{[15:14]} & rw & \texttt{0x0} & HDMI Output 2 S-bit mode \\\hline
    SBIT\_MODE3 & \texttt{0x13} & \texttt{[17:16]} & rw & \texttt{0x0} & HDMI Output 3 S-bit mode \\\hline
    SBIT\_MODE4 & \texttt{0x13} & \texttt{[19:18]} & rw & \texttt{0x0} & HDMI Output 4 S-bit mode \\\hline
    SBIT\_MODE5 & \texttt{0x13} & \texttt{[21:20]} & rw & \texttt{0x0} & HDMI Output 5 S-bit mode \\\hline
    SBIT\_MODE6 & \texttt{0x13} & \texttt{[23:22]} & rw & \texttt{0x0} & HDMI Output 6 S-bit mode \\\hline
    SBIT\_MODE7 & \texttt{0x13} & \texttt{[25:24]} & rw & \texttt{0x0} & HDMI Output 7 S-bit mode \\\hline
    \end{tabularx}
    \vspace{5mm}


    \noindent
    \subsection*{\textcolor{parentcolor}{\textbf{FPGA.CONTROL.CNT\_SNAP}}}

    \vspace{4mm}
    \noindent
    Control the global counter snapshot
    \noindent

    \keepXColumns
    \begin{tabularx}{\linewidth}{ | l | l | r | c | l | X | }
    \hline
    \textbf{Node} & \textbf{Adr} & \textbf{Bits} & \textbf{Perm} & \textbf{Def} & \textbf{Description} \\\hline
    \nopagebreak
    PULSE & \texttt{0x14} & \texttt{[0:0]} & w & Pulse & Pulse to take a counter snapshot \\\hline
    DISABLE & \texttt{0x15} & \texttt{[1:1]} & rw & \texttt{0x1} & 0=enable snapshots (counters freeze synchronously and need a snapshot to update) \\\hline
    \end{tabularx}
    \vspace{5mm}


    \noindent
    \subsection*{\textcolor{parentcolor}{\textbf{FPGA.CONTROL.DNA}}}

    \vspace{4mm}
    \noindent
    57 Bit FPGA-specific device identifier
    \noindent

    \keepXColumns
    \begin{tabularx}{\linewidth}{ | l | l | r | c | l | X | }
    \hline
    \textbf{Node} & \textbf{Adr} & \textbf{Bits} & \textbf{Perm} & \textbf{Def} & \textbf{Description} \\\hline
    \nopagebreak
    DNA\_LSBS & \texttt{0x17} & \texttt{[31:0]} & r & \texttt{} & Device DNA bits 31 downto 0 \\\hline
    DNA\_MSBS & \texttt{0x18} & \texttt{[24:0]} & r & \texttt{} & Device DNA bits 56 downto 32 \\\hline
    \end{tabularx}
    \vspace{5mm}


    \noindent
    \subsection*{\textcolor{parentcolor}{\textbf{FPGA.CONTROL}}}

    \vspace{4mm}
    \noindent
    Implements various control and monitoring functions of the Optohybrid
    \noindent

    \keepXColumns
    \begin{tabularx}{\linewidth}{ | l | l | r | c | l | X | }
    \hline
    \textbf{Node} & \textbf{Adr} & \textbf{Bits} & \textbf{Perm} & \textbf{Def} & \textbf{Description} \\\hline
    \nopagebreak
    UPTIME & \texttt{0x19} & \texttt{[19:0]} & r & \texttt{} & Uptime in seconds \\\hline
    USR\_ACCESS & \texttt{0x20} & \texttt{[31:0]} & r & \texttt{} & Git hash read from USR\_ACCESS field \\\hline
    \end{tabularx}
    \vspace{5mm}


    \noindent
    \subsection*{\textcolor{parentcolor}{\textbf{FPGA.CONTROL.HOG}}}

    \keepXColumns
    \begin{tabularx}{\linewidth}{ | l | l | r | c | l | X | }
    \hline
    \textbf{Node} & \textbf{Adr} & \textbf{Bits} & \textbf{Perm} & \textbf{Def} & \textbf{Description} \\\hline
    \nopagebreak
    GLOBAL\_DATE & \texttt{0x21} & \texttt{[31:0]} & r & \texttt{} & HOG Global Date \\\hline
    GLOBAL\_TIME & \texttt{0x22} & \texttt{[31:0]} & r & \texttt{} & HOG Global Time \\\hline
    GLOBAL\_VER & \texttt{0x23} & \texttt{[31:0]} & r & \texttt{} & HOG Global Version \\\hline
    GLOBAL\_SHA & \texttt{0x24} & \texttt{[31:0]} & r & \texttt{} & HOG Global SHA \\\hline
    TOP\_SHA & \texttt{0x25} & \texttt{[31:0]} & r & \texttt{} & HOG Top SHA \\\hline
    TOP\_VER & \texttt{0x26} & \texttt{[31:0]} & r & \texttt{} & HOG Top Version \\\hline
    HOG\_SHA & \texttt{0x27} & \texttt{[31:0]} & r & \texttt{} & HOG SHA \\\hline
    HOG\_VER & \texttt{0x28} & \texttt{[31:0]} & r & \texttt{} & HOG Version \\\hline
    OH\_SHA & \texttt{0x29} & \texttt{[31:0]} & r & \texttt{} & OH SHA \\\hline
    OH\_VER & \texttt{0x2a} & \texttt{[31:0]} & r & \texttt{} & OH Version \\\hline
    FLAVOUR & \texttt{0x2b} & \texttt{[31:0]} & r & \texttt{} & Flavor \\\hline
    \end{tabularx}
    \vspace{5mm}


% END: ADDRESS_TABLE :: DO NOT EDIT

\end{document}
